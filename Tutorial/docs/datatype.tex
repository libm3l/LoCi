\chapter{Datatypes}
%\appendix{Datatypes }
%
\section {Introduction}
For the purpose of I/O and communication, datatypes can be considered
as abstract representations of the state of an object.  Usually this
is represented by the memory locations where data is stored.  In the
Loci framework, we gave to specify how to save and restore the state
of new object types explicitly to facilitate interprocessor
communication in heterogeneous environments or portable file I/O.
This information is provided to Loci using the traits mechanism.

In Loci, the datatype information is encapsulated in {\tt
  data\_schema\_traits} template class.  A user will need to provide a
specialized template for his/her own datatypes before they can be
used with Loci containers such as {\tt store}.
%
\section {Classification of Datatypes} 

Datatypes could be classified according to the their relationship
between computer memory representation and data which they hold.  The
basic distinction depends on the way C++ stores the object in memory.
If the object is represented in a continuous segment of memory with a
fixed layout, then then the memory layout is consistent with the
schemas used to send messages or write data to files.  If, instead the
objects data is scattered through memory or has a size that is
determined at run time, then the object must be serialized before it
can be sent as a message or placed in file storage.  An object whose
memory layout is compatible with the contiguous memory schema uses an
{\tt IDENTITY\_SCHEMA\_CONVERTER} for serialization.  Other objects
must also specify a schema converter.  The specifics of this will be
specified in the below examples:
%
\begin {verbatim}
typedef struct My_Type_t {
    int       iScalar;
    float     fScalar;
    char      cScalar;
    double    dScalar;
} My_Type1;

typedef struct My_Type_t {
    int        iScalar;
    float      fScalar;:
    char       cScalar;
    double    *dScalar;
} My_Type2;
\end{verbatim}
%
For {\em My\_Type1} C/C++ guarantees that the members of a structure
are stored in contiguous memory locations with some padding, so the
bitwise copy of {\em My\_Type1} will pack the data correctly, whereas
for {\em My\_Type2} the address of {\em dScalar} and not the data will
be copied. In the first case, the memory schema and the file schema is
handled by an {\bf Identity Function} called the {\tt
  IDENTITY\_SCHEMA\_CONVERTER }

In the second case, we need to explicitly convert between memory
schema and the file schema by specifying {\tt USER\_DEFINED\_CONVERTER}.

It should be mentioned here, that pointers are not the only source of
difference between these two datatype. STL container like vector,
list, queues, maps and virtual classes, all would need user to
serialize the state they contains, therefore, they also fall under the
category of user defined converter type.


\section {Predefined Datatypes in Loci}
Loci provides some of the frequently used datatypes so that user need
not write for themselves.These are
%
%
\begin{enumerate}
\item {\bf Atomic Datatype}
As the name implies, these datatypes are indivisible types. These type
are supported by all machines. These are building blocks for all other
datatypes. All native C/C++ datatypes are atomic datatypes
in Loci. The following table provides all the atomic datatypes support
by the Loci.
%
\begin{center}
\begin{tabular} [h] {|l|l|} \hline
Loci Datatype & Native C++ Datatype \\ \hline \hline
BOOL & bool \\ \hline
CHAR & signed char \\\hline
UNSIGNED\_CHAR & unsigned char \\\hline
SHORT & short \\\hline
UNSIGNED\_SHORT & unsigned short \\\hline
INT & int \\\hline
UNSIGNED & unsigned \\\hline
LONG & long \\\hline
UNSIGNED\_LONG & unsigned long \\\hline
FLOAT & float \\\hline
DOUBLE & double \\\hline
\end{tabular}
\end{center}
\item {\bf Array class}
Loci, provides template version of constant size array
\begin{verbatim}
template <class T, unsigned int n>
class Array {
   public:
   .
   .
   private:
      T  x[n];
};
\end{verbatim}
\item {\bf Vector class}
Loci, has 2D/3D vesions of vector class(mathematical vectors)
\begin{verbatim}
 template<T>
 struct vector3d {
      T x,y,z ;
 }

 template<T>
 struct vector2d {
      T x,y ;
 }
\end{verbatim}
\item {\bf STL containers parameterized by types using the {\tt
      IDENTITY\_SCHEMA\_CONVERTER}} 

All standard STL containers with predefined identity schema types are supported.
\begin{enumerate}
\item vector
\item list
\item queue
\item set
\item map
\end{enumerate}
\end{enumerate}
\section {Creating your own compound datatypes}
%
Loci, uses {\tt data\_schema\_traits} template class to determine
the datatype of an object. The generic behavior of template class is
not suitable for identifying or creating new datatype, so we use {\bf
  template Specialization } technique to customize or create new
datatypes . This {\tt data\_schema\_traits} class has one static
member function {\tt get\_type()} in which user specifies the
information about new datatype. A general skeleton for creating new
datatype look as follows
\begin{verbatim}
namespace Loci {
   // Skelton for datatype having identity schema
   template <>
   struct data_schema_traits <My_New_Type1 > {
         typedef IDENTITY_CONVERTER Schema_Converter;
         static DatatypeP get_type() {
                CompoundDatatypeP cmpd = CompoundFactory(My_New_Type1());
                LOCI_INSERT_TYPE(cmpd, My_New_Type1, member);
                   .
                   .
                return DatatypeP(cmpd);
         }
   };
}
\end{verbatim}
%
Where {\bf CompoundFactory} is one of the software design pattern for creating a 
new compound datatype object. Since the allocation of this object is done inside
the functions, the question will always arise, who is responsible for deleting
the object ? To destroy the objects when they are not needed, we use reference counting and the 
{\tt CompoundDatatypeP} class stands for {\em Reference Counting} version of
{\tt CompoundDatatype} class.
%
\par If your C/C++ structures contains members of predefined Loci datatypes, then it
is fairly easy to create corresponding Loci datatype. For example
\begin{verbatim}
 typedef struct My_Compound_Type_t {
        float                       fScalar;      
        vector3d<double>            vect3d;
        Array<double,2>             array1d;
        Array<Array<double,2>,4>    array2d;
 } My_Compound_Type;

 namespace Loci {
  template <> struct data_schema_traits <My_Compound_Type > {
     typedef IDENTITY_CONVERTER Schema_Converter;
     static DatatypeP get_type() {
         CompoundDatatypeP cmpd = CompoundFactory(My_Compound_Type());
         LOCI_INSERT_TYPE(cmpd, My_Compound_Type, fScalar);
         LOCI_INSERT_TYPE(cmpd, My_Compound_Type, vect3d);
         LOCI_INSERT_TYPE(cmpd, My_Compound_Type, array1d);
         LOCI_INSERT_TYPE(cmpd, My_Compound_Type, array2d);
         return DatatypeP(cmpd);
      }
    };
  }
\end{verbatim}
\section{Creating User Defined Datatype}
\par As explained earlier, whenever the memory allocation in any object is
not contiguous, it is the users responsibility to serialize the
data contained in the objects.  The skeleton of user defined schema type
will look as follows
\begin{verbatim}
   // Skelton for datatype having user defined schema
   template <>
   struct data_schema_traits <My_New_Type2 > {
        typedef USER_DEFINED_CONVERTER Schema_Converter;
        typedef char  Converter_Base_Type;
        typedef MyObject_SchemaConverter   Converter_Type;
   };
\end{verbatim}
\par {\em Note : {\tt Converter\_Base\_Type} could be any datatype with identity schema}
\par Following steps must be taken in order to define your own datatype
for Loci
\begin{enumerate} 
\item Specialize the {\tt data\_schema\_traits} class
\begin{itemize}
\item Specialize {\tt data\_schema\_traits} template class with your
class
\item \par declare in {\tt data\_schema\_traits} class
\begin{center}
{\tt typedef USER\_DEFINED\_CONVERTER Schema\_Converter}
\end{center}
\item specify what datatype will be used in conversion. This should be 
datatype with identity schema defined which means, that we can use any 
valid Loci atomic or compound datatype.
\item specify the class which has the responsibility of conversion (
Serialize class )
\end{itemize}
\item Specifying Serialize class
\begin{itemize}
\item Specify object reference in the constructor.
\item {\tt getSize()} member function returns the number of atomic
  datatypes used in this object.(It is not the size of the object in
  bytes)
\item {\tt getState()} member function gets the state of an
object into a contiguous buffer.
\item {\tt setState()} member function sets the state of an
object from a contiguous buffer.
\end{itemize}
\item Overload input/output stream functions.
Both atomic and compound datatypes have already been overloaded with
input/output streams in Loci.{\bf It is required that these function 
are overloaded even if the a user doesn't have intention of
using them.}
\end{enumerate}
%
Now we shall give one simple example to show how things work. We define
one structure with STL list inside it. Since list may not have contiguous
memory, we define it is defined with user defined schema
\begin{verbatim}

  namespace Loci {
  /////////////////////////////////////////////////////////////////////////////////
  // This is an example of conventional C/C++ structure 
  /////////////////////////////////////////////////////////////////////////////////

     struct My_Type {
      list<int>  alist;
      friend ostream& operator << (ostream &, const My_Type &);
      friend istream& operator >> (istream &, My_Type &);
  };

  //------------------------------------------------------------------------------/
  class My_Type_SchemaConverter;    // Forward Declaration of class
  //------------------------------------------------------------------------------/
  // Specialize the data_schema_traits class with "My_Type" class
  //------------------------------------------------------------------------------/

  template <>
  struct data_schema_traits<My_Type> {
    // This class has user defined schema
    typedef USER_DEFINED_CONVERTER Schema_Converter ;

    // Since list contains "int" we use it directly for our converion
    typedef int Converter_Base_Type ;
   
    // Here we specify the class used for serialization/deserialization 
    // purpose
    typedef My_Type_SchemaConverter Converter_Type ;
  };

  //------------------------------------------------------------------------------/
  // Define a class which has the responsibity of serialization and deserialization 
  // of "My_Type" class
  //------------------------------------------------------------------------------/

  class My_Type_SchemaConverter {
    // For the schema converter, we always store a reference to the object
    // we are converting schmata for.
    My_Type &RefObj ;
    public:
        explicit My_Type_SchemaConverter(My_Type &new_obj): RefObj(new_obj) {}
        //
        // This member function returns number of elements of type defined
        // in Converter_Base_Type. It is not the size in bytes.
        //
        int getSize() const {
            return RefObj.alist.size() ;
        }
      
        // Get the state of an object "RefObj" into an array and also size of
        // array. This is a serialization step. 
        void getState(int *buf, int &size) {
             size = getSize() ;
             int ii=0;

             list<int> :: const_iterator ci;
             list<int> :: const_iterator begin = RefObj.alist.begin();
             list<int> :: const_iterator end   = RefObj.alist.end();
             for(ci = begin; ci != end; ++ci) 
                 buf[ii++] = *ci;
        }
        //
        // From a given array, construct the object. This is "Deserialization Step"
        //
        void setState(int *buf, int size) {
             RefObj.alist.clear();
             list<int> :: iterator ci;
             for(int i=0;i<size;++i)
                 RefObj.alist.push_back(buf[i]);
        }
  };
  }
  //------------------------------------------------------------------------------/
\end{verbatim}
%
%
\section{Inner Details about Compound Datatype}
Compound datatypes are similar to structures in C/C++. These datatypes
are a collection of heterogeneous atomic or fixed sized array datatypes. 
Every member of these datatype has a unique name within the datatype and 
they occupy  non-overlapping memory locations.
%
%
\par In Loci, this datatype is declared in {\tt CompoundType}
class. The corresponding counted pointer class is {\tt
CompoundDatatypeP}.
%
\par A new member can be inserted into the new datatype in either way
\begin{itemize}
\item Using member function of CompoundType class
\begin{center}
insert( member\_name, offsetof(type, member-designator), member\_datatype);
\end{center}
where {\tt offsetof} is a standard C/C++ function which provides offset of any
member (designated by member-designator) in C/C++ structure (designated by type).
{\em member\_datatype} could be any valid Loci datatype.
%
\item Using predefined macro
\begin{center}
LOCI\_INSERT\_TYPE( compound\_object, compound\_class, insert\_member);
\end{center}
\par where {\em compound\_object} is the compound datatype for {\em compound\_class}
and {\em insert\_member} is the required member of {\em compound\_class} which
is inserted into new datatype. 
\par In order to use the macro {\em insert\_member} must be a first class object
and and its own type should be identified by {\tt data\_schema\_traits} class.
\end{itemize}
\par In the following sections we shall gives some examples of
creating different Loci datatypes.
%
\subsection {Creating compound datatype with only atomic datatypes}
The following is a very simple C/C++ structure, which contains only native datatypes.
For this structure, we would like to create Loci datatype, which is also described
below.
\begin{verbatim}
 typedef struct My_Compound_Type_t {
        int        iScalar;
        float      fScalar;
        char       cScalar;
        double     dScalar;
 } My_Compound_Type;
\end{verbatim}
\begin{verbatim}
  // data_scheme_traits should be defined in Loci namespace
 namespace Loci {
   // Specialize the class with the new class
   template <> 
   struct data_schema_traits <My_Compound_Type > {

        // Specify that ideneity Schema will be used for this class
        typedef IDENTITY_CONVERTER Schema_Converter;

        // define the member function
        static DatatypeP get_type() {
              // Create a new product "cmpd" from the factory pattern 
              CompoundDatatypeP cmpd = CompoundFactory(My_Compound_Type());
              // Insert a new member into the new compound datatype
              LOCI_INSERT_TYPE(cmpd, My_Compound_Type, iScalar);
              LOCI_INSERT_TYPE(cmpd, My_Compound_Type, fScalar);
              LOCI_INSERT_TYPE(cmpd, My_Compound_Type, cScalar);
              LOCI_INSERT_TYPE(cmpd, My_Compound_Type, dScalar);
             // return pointer to the base class
             return DatatypeP(cmpd);
         }
      };
  }
\end{verbatim}
\subsection{Creating compound datatype with arrays}
In the following example, we have inserted a two dimensional array
into the structure and define corresponding Loci datatype.
\begin{verbatim}
 namespace Loci {
     typedef struct My_Compound_Type_t {
         int     iScalar;
         float   fScalar;
         double  dScalar;
         Array<double,10>          dArray1D;
         Array<Array<double,3>,5>  dArray2D;
     } My_Compound_Type;
    
     template<>
     struct data_schema_traits<My_Compound_Type> {
          typedef IDENTITY_CONVERTER Schema_Converter ;
          static DatatypeP get_type() {
            CompoundDatatypeP ct = CompoundFactory(My_Compound_Type()) ;
            LOCI_INSERT_TYPE(ct,My_Compound_Type, iScalar) ;
            LOCI_INSERT_TYPE(ct,My_Compound_Type, fScalar) ;
            LOCI_INSERT_TYPE(ct,My_Compound_Type, dScalar) ;
            LOCI_INSERT_TYPE(ct,My_Compound_Type, dArray1D) ;
            LOCI_INSERT_TYPE(ct,My_Compound_Type, dArray2D) ;
            return DatatypeP(ct) ;
          }
      } ;
 }
\end{verbatim}
\subsection{Creating compound datatype with nested compound
datatypes}
In this example, we would demonstrate that any member with valid
datatype could be inserted into compound datatype in an hierarchal fashion. There
is no restriction on number of levels used to define a datatype.
\begin{verbatim}
namespace Loci {
   struct Velocity {
     Array<double,3>  comp;
   };
   // Define traits for "Velocity" structure
   template <>
   struct data_schema_traits<Velocity> {
     typedef IDENTITY_CONVERTER Schema_Converter ;
     static DatatypeP get_type() {
       Velocity v;
       return getLociType(v.comp);
     }
   };
   // A structure contains another structure
   struct CellAttrib {
     int      local_id;
     double   density;
     double   pressure;
     Velocity vel;
   };
 
   // Define traits for "CellAttrib" class. Notice that "vel" is a structure
   // and since its type is already defined, we can insert it similar to other
   // members.
   template <>
   struct data_schema_traits<CellAttrib> {
     typedef IDENTITY_CONVERTER Schema_Converter ;
     static DatatypeP get_type() {
       CompoundDatatypeP  cmpd = CompoundFactory( CellAttrib() );
       LOCI_INSERT_TYPE(cmpd, CellAttrib, local_id );
       LOCI_INSERT_TYPE(cmpd, CellAttrib, density  );
       LOCI_INSERT_TYPE(cmpd, CellAttrib, pressure );
       LOCI_INSERT_TYPE(cmpd, CellAttrib, vel );
       return DatatypeP(cmpd);
     }
   };
 }
\end{verbatim}

\section{Array Datatype}
Array datatype consists of homogeneous collection of both compound and 
atomic datatypes. We can defined array datatypes for standard C/C++
arrays. In order to create array datatype, a user need to provide
\begin{itemize}
\item the rank of the array, i.e. number of dimensions 
\par {\em Note At present, Loci can support arrays with maximum rank of
4. This limitation comes from HDF5 library. If the user wants higher ranked
arrays, using {\bf Array class} is one solution}
\item the size of each dimension
\item the datatype of each element of the array
\end{itemize}
\par In Loci, this datatype is declared in {\tt ArrayType} class. The 
corresponding counted pointer class is {\tt ArrayDatatypeP}.
%
\begin{itemize}
\item Create 1 D dimensional array datatype
\begin{verbatim}
dims[0]= 100;
ArrayType  atype(Loci::DOUBLE, 1, dims);
\end{verbatim}
%
\item Create 2 D dimensional array datatype
\begin{verbatim}
dims[0] = 10;
dims[1] = 20;
ArrayType  atype(Loci::DOUBLE, 2, dims);
\end{verbatim}
%
\item Create 3 D dimensional array datatype
\begin{verbatim}
dims[0] = 10;
dims[1] = 20;
dims[2] = 30;
ArrayType  atype(Loci::DOUBLE, 3, dims);
\end{verbatim}
%
\item Create 4 D dimensional array datatype
\begin{verbatim}
dims[0] = 10;
dims[1] = 20;
dims[2] = 30;
dims[3] = 40;
ArrayType  atype(Loci::DOUBLE, 4, dims);
\end{verbatim}
\end{itemize}
%
%
\par For example
\begin{verbatim}
typedef struct My_Compound_Type_t {
    int       iScalar;
    float     fScalar;
    char      cScalar;
    double    dScalar[10][5][2];
}My_Compound_Type;
\end{verbatim}
%
This datatype has multidimensional array, which is not first class
objects. We can use {\tt ArrayDatatype} to specify the Loci Datatype as
\begin{verbatim}
int     rank  = 3;
int     dim[] = {10, 5, 2};
int     sz    = 100*sizeof(double);

My_Compound_Type     type;
CompoundDatatypeP cmpd    = CompoundFactory(My_Compound_Type());
DatatypeP         atom    = getLociType(type.dScalar[0][0][0]);
ArrayDatatypeP    array_t = ArrayFactory(atom, sz, rank, dim);
cmpd->insert("dScalar", offsetof(My_Compound_Type, dScalar), DatatypeP(array_t));
\end{verbatim}

This is definitely cumbersome, Instead of using array in this way, if
we had used
\begin{verbatim}
typedef  Array < double, 10 > Array1D;
typedef  Array < Array1D, 5 > Array2D;
typedef  Array < Array2D, 2 > Array3D;
typedef struct My_Compound_Type_t {
   int             iScalar;
   float           fScalar;
   char            cScalar;
   Array3D         dScalar;
} My_Compound_Type;
\end{verbatim}
then we can use {\tt LOCI\_INSERT\_TYPE} macro to insert {\tt
dScalar} into new compound datatype.

%
